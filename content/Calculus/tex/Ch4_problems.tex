\documentclass[12pt]{article}

\usepackage[margin=1in]{geometry}
\usepackage{amsmath,amssymb}
\usepackage{enumitem}
\usepackage{booktabs,array}
\usepackage{tikz}
\usepackage{pgfplots}
\pgfplotsset{compat=1.17}

\setlength{\parskip}{0.5em}
\setlength{\parindent}{0pt}

\begin{document}

{\LARGE \textbf{Practice Exercises}}

\bigskip
\noindent\textbf{PART A. NO CALCULATOR—DIRECTIONS:} Answer these questions \emph{without} using your calculator.

\begin{enumerate}[label=\textbf{A\arabic*.}]

% --- A1
\item The slope of the curve $y^{3}-xy^{2}=4$ at the point where $y=2$ is
\begin{enumerate}[label=(\Alph*)]
\item $-2$ \item $-\dfrac12$ \item $\dfrac12$ \item $2$
\end{enumerate}

% --- A2
\item The slope of the curve $y^{2}-xy-3x=1$ at the point $(0,-1)$ is
\begin{enumerate}[label=(\Alph*)]
\item $-1$ \item $-2$ \item $1$ \item $2$
\end{enumerate}

% --- A3
\item An equation of the tangent to the curve $y=x\sin x$ at the point $\bigl(\tfrac{\pi}{2},\tfrac{\pi}{2}\bigr)$ is
\begin{enumerate}[label=(\Alph*)]
\item $y=x-\pi$ \item $y=\dfrac{\pi}{2}$ \item $y=\pi-x$ \item $y=x$
\end{enumerate}

% --- A4
\item The tangent to the curve $y=xe^{-x}$ is horizontal when $x$ is equal to
\begin{enumerate}[label=(\Alph*)]
\item $0$ \item $1$ \item $-1$ \item $\dfrac{1}{e}$
\end{enumerate}

% --- A5
\item The minimum value of the slope of the curve $y=x^{5}+x^{3}-2x$ is
\begin{enumerate}[label=(\Alph*)]
\item $2$ \item $6$ \item $-2$ \item $-6$
\end{enumerate}

% --- A6
\item An equation of the tangent to the hyperbola $x^{2}-y^{2}=12$ at the point $(4,2)$ on the curve is
\begin{enumerate}[label=(\Alph*)]
\item $x-2y+6=0$ \item $y=2x$ \item $y=2x-6$ \item $y=\dfrac{x}{2}$
\end{enumerate}

% --- A7
\item A tangent to the curve $y^{2}-xy+9=0$ is vertical when
\begin{enumerate}[label=(\Alph*)]
\item $y=0$ \item $y=\pm\sqrt{3}$ \item $y=\dfrac12$ \item $y=\pm 3$
\end{enumerate}

% --- A8
\item The volume of a sphere is given by $V=\dfrac{4}{3}\pi r^{3}$. Use a tangent line to approximate the increase in volume (in $\text{in}^{3}$) when the radius increases from $3$ to $3.1$ inches.
\begin{enumerate}[label=(\Alph*)]
\item $\dfrac{0.04\pi}{3}$ \item $0.04\pi$ \item $1.2\pi$ \item $3.6\pi$
\end{enumerate}

% --- A9
\item When $x=3$, the equation $2x^{2}-y^{3}=10$ has the solution $y=2$. Using the tangent line to the curve, approximate $y$ when $x=3.04$.
\begin{enumerate}[label=(\Alph*)]
\item $1.6$ \item $1.96$ \item $2.04$ \item $2.4$
\end{enumerate}

\end{enumerate}

\textbf{Challenge.}

\begin{enumerate}[label=\textbf{A\arabic*.},resume]

% --- A10
\item If the side $e$ of a square is increased by $1\%$, then the area is increased approximately by
\begin{enumerate}[label=(\Alph*)]
\item $0.02e$ \item $0.02e^{2}$ \item $0.01e^{2}$ \item $0.01e$
\end{enumerate}

% --- A11
\item The edge of a cube has length $10$ in., with a possible error of $1\%$. The possible error (in $\text{in}^{3}$) in the volume of the cube is
\begin{enumerate}[label=(\Alph*)]
\item $1$ \item $3$ \item $10$ \item $30$
\end{enumerate}

% --- A12
\item The function $f(x)=x^{4}-4x^{2}$ has
\begin{enumerate}[label=(\Alph*)]
\item one local minimum and two local maxima
\item one local minimum and no local maximum
\item no local minimum and one local maximum
\item two local minima and one local maximum
\end{enumerate}

% --- A13
\item The number of inflection points on the graph of $f(x)=x^{4}-4x^{2}$ is
\begin{enumerate}[label=(\Alph*)]
\item $0$ \item $1$ \item $2$ \item $3$
\end{enumerate}

% --- A14
\item The maximum value of the function $y=-4\sqrt{\,2-x\,}$ is
\begin{enumerate}[label=(\Alph*)]
\item $0$ \item $-4$ \item $-2$ \item $2$
\end{enumerate}

% --- A15
\item The total number of local maximum and minimum points of a function whose derivative is $f'(x)=x(x-3)^{2}(x+1)^{4}$ (for all $x$) is
\begin{enumerate}[label=(\Alph*)]
\item $0$ \item $1$ \item $2$ \item $3$
\end{enumerate}

% --- A16 (figure with four curves)
\item For which curve shown below are both $f'$ and $f''$ negative?
\begin{center}
\begin{tikzpicture}[scale=0.9]
  % col / row / label / function in terms of \x
  \foreach \col/\row/\lab/\fx in {
    0/0/A/{-0.3*(\x+1)^2 - 0.6*\x - 0.2},
    6/0/B/{ 0.2*(\x-0.4)^2 + 0.3*\x - 0.7},
    0/-3.3/C/{ 0.25*(\x-0.6)^2 + 0.2*(\x-0.6) + 0.1},
    6/-3.3/D/{-0.25*(\x-0.2)^2 + 0.6 - 0.2*\x}
  }{
    \begin{scope}[shift={(\col,\row)}]
      \draw[->] (-1.6,0)--(1.6,0);
      \draw[->] (0,-1.2)--(0,1.2);
      \draw[domain=-1.4:1.4,smooth,thick] plot (\x,{\fx});
      \node at (0,-1.45) {(\lab)};
    \end{scope}
  }
\end{tikzpicture}
\end{center}

\begin{enumerate}[label=(\Alph*)]
\item Curve (A) \item Curve (B) \item Curve (C) \item Curve (D)
\end{enumerate}

% --- A17
\item For which curve shown in Question A16 is $f''>0$ but $f'<0$?
\begin{enumerate}[label=(\Alph*)]
\item Curve (A) \item Curve (B) \item Curve (C) \item Curve (D)
\end{enumerate}

\end{enumerate}

\medskip\noindent\textit{In Questions A18–A21, the position of a particle moving along a horizontal line is given by $s=t^{3}-6t^{2}+12t-8$.}

\begin{enumerate}[label=\textbf{A\arabic*.},resume]

\item The object is moving to the right for
\begin{enumerate}[label=(\Alph*)]
\item $t<2$ \item all $t$ except $t=2$ \item all $t$ \item $t>2$
\end{enumerate}

\item The minimum value of the speed is
\begin{enumerate}[label=(\Alph*)]
\item $0$ \item $1$ \item $2$ \item $3$
\end{enumerate}

\item The acceleration is positive
\begin{enumerate}[label=(\Alph*)]
\item when $t>2$ \item for all $t$, $t\ne 2$ \item when $t<2$ \item for $1<t<2$
\end{enumerate}

\item The speed of the particle is decreasing for
\begin{enumerate}[label=(\Alph*)]
\item $t<2$ \item all $t$ \item $t<1$ or $t>2$ \item $t>2$
\end{enumerate}

\end{enumerate}

\medskip\noindent\textit{In Questions A22–A24, a particle moves along a horizontal line with position $s=t^{4}-6t^{3}+12t^{2}+3$.}

\begin{enumerate}[label=\textbf{A\arabic*.},resume]

\item The particle is at rest when $t$ is equal to
\begin{enumerate}[label=(\Alph*)]
\item $1$ or $2$ \item $0$ \item $\dfrac{9}{4}$ \item $0,1,$ or $2$
\end{enumerate}

\item The velocity $v$ is increasing when
\begin{enumerate}[label=(\Alph*)]
\item $t>1$ \item $1<t<2$ \item $t<2$ \item $t<1$ or $t>2$
\end{enumerate}

\item The speed of the particle is increasing for
\begin{enumerate}[label=(\Alph*)]
\item $0<t<1$ or $t>2$ \item $1<t<2$ \item $t<2$ \item $t<0$ or $t>2$
\end{enumerate}

% --- A25
\item The displacement from the origin of a particle moving on a line is $s=t^{4}-4t^{3}$. The maximum displacement during the time interval $-2\le t\le 4$ is
\begin{enumerate}[label=(\Alph*)]
\item $27$ \item $3$ \item $48$ \item $16$
\end{enumerate}

% --- A26
\item If a particle moves along a line according to $s=t^{5}+5t^{4}$, then the number of times it reverses direction is
\begin{enumerate}[label=(\Alph*)]
\item $0$ \item $1$ \item $2$ \item $3$
\end{enumerate}

\end{enumerate}

\medskip\noindent\textit{*In Questions A27–A30, $\mathbf R=\langle 3\cos(\tfrac{\pi}{3}t),\,2\sin(\tfrac{\pi}{3}t)\rangle$ is the (position) vector $\langle x,y\rangle$ from the origin to a moving point $P(x,y)$ at time $t$.}

\begin{enumerate}[label=\textbf{A\arabic*.},resume]

\item A single equation in $x$ and $y$ for the path of the point is
\begin{enumerate}[label=(\Alph*)]
\item $x^{2}+y^{2}=13$
\item $9x^{2}+4y^{2}=36$
\item $2x^{2}+3y^{2}=13$
\item $4x^{2}+9y^{2}=36$
\end{enumerate}

\item When $t=3$, the speed of the particle is
\begin{enumerate}[label=(\Alph*)]
\item $\dfrac{2\pi}{3}$ \item $2$ \item $3$ \item $\dfrac{\sqrt{13}}{3}\pi$
\end{enumerate}

\item The magnitude of the acceleration when $t=3$ is
\begin{enumerate}[label=(\Alph*)]
\item $2$ \item $\dfrac{\pi^{2}}{3}$ \item $3$ \item $\dfrac{2\pi^{2}}{9}$
\end{enumerate}

\item At the point where $t=\tfrac12$, the slope of the curve along which the particle moves is
\begin{enumerate}[label=(\Alph*)]
\item $\dfrac{2\sqrt3}{9}$ \item $-\dfrac{\sqrt3}{2}$ \item $\dfrac{2}{\sqrt3}$ \item $-\dfrac{2\sqrt3}{3}$
\end{enumerate}

% --- A31
\item A balloon is being filled with helium at the rate of $4\ \text{ft}^{3}/\text{min}$. The rate (in $\text{ft}^{2}/\text{min}$) at which the surface area is increasing when the volume is $\dfrac{32\pi}{3}\ \text{ft}^{3}$ is
\begin{enumerate}[label=(\Alph*)]
\item $4\pi$ \item $2$ \item $4$ \item $1$
\end{enumerate}

% --- A32
\item A circular conical reservoir, vertex down, has depth $20$ ft and radius at the top $10$ ft. Water is leaking out so that the surface is falling at the rate of $\tfrac12$ ft/hr. The rate (in ft$^{3}$/hr) at which the water is leaving the reservoir when the water is $8$ ft deep is
\begin{enumerate}[label=(\Alph*)]
\item $4\pi$ \item $8\pi$ \item $\dfrac{1}{4\pi}$ \item $\dfrac{1}{8\pi}$
\end{enumerate}

% --- A33
\item A local minimum value of the function $y=\dfrac{e^{x}}{x}$ is
\begin{enumerate}[label=(\Alph*)]
\item $\dfrac{1}{e}$ \item $1$ \item $-1$ \item $e$
\end{enumerate}

\end{enumerate}

\textbf{Challenge.}

\begin{enumerate}[label=\textbf{A\arabic*.},resume]
% --- A34
\item The area of the largest rectangle that can be drawn with one side along the $x$-axis and two vertices on the curve $y=e^{-x^{2}}$ is
\begin{enumerate}[label=(\Alph*)]
\item $\dfrac{\sqrt{2}}{e}$ \item $\sqrt{2e}$ \item $\dfrac{2}{e}$ \item $\dfrac{1}{\sqrt{2}\,e}$
\end{enumerate}

% --- A35
\item A line with a negative slope is drawn through the point $(1,2)$ forming a right triangle with the positive $x$- and $y$-axes. The slope of the line forming the triangle of least area is
\begin{enumerate}[label=(\Alph*)]
\item $-1$ \item $-2$ \item $-3$ \item $-4$
\end{enumerate}

% --- A36
\item The point(s) on the curve $x^{2}-y^{2}=4$ closest to the point $(6,0)$ is (are)
\begin{enumerate}[label=(\Alph*)]
\item $(2,0)$ \item $\bigl(\sqrt{5},\pm 1\bigr)$ \item $\bigl(3,\pm\sqrt{5}\bigr)$ \item $\bigl(\sqrt{13},\pm\sqrt{3}\bigr)$
\end{enumerate}

% --- A37
\item The sum of the squares of two positive numbers is $200$; their minimum product is
\begin{enumerate}[label=(\Alph*)]
\item $100$ \item $20$ \item $0$ \item there is no minimum
\end{enumerate}

% --- A38
\item The first-quadrant point on the curve $y^{2}x=18$ that is closest to the point $(2,0)$ is
\begin{enumerate}[label=(\Alph*)]
\item $(2,3)$ \item $(6,\sqrt{3})$ \item $(3,\sqrt{6})$ \item $\bigl(1,3\sqrt{2}\bigr)$
\end{enumerate}

% --- A39
\item If $h$ is a small negative number, then the local linear approximation for $\sqrt[3]{27+h}$ is
\begin{enumerate}[label=(\Alph*)]
\item $3+\dfrac{h}{27}$ \item $3-\dfrac{h}{27}$ \item $\dfrac{h}{27}$ \item $-\dfrac{h}{27}$
\end{enumerate}

% --- A40
\item If $f(x)=xe^{-x}$, then at $x=0$
\begin{enumerate}[label=(\Alph*)]
\item $f$ is increasing \item $f$ is decreasing \item $f$ has a relative maximum \item $f$ has a relative minimum
\end{enumerate}

% --- A41
\item A function $f$ has a derivative for each $x$ such that $|x|<2$ and has a local minimum at $(2,-5)$. Which statement below must be true?
\begin{enumerate}[label=(\Alph*)]
\item $f'(2)=0$ \item $f$ exists at $x=2$ \item $f'(x)<0$ if $x<2$, $f'(x)>0$ if $x>2$ \item none of the preceding is necessarily true
\end{enumerate}

% --- A42
\item The height of a rectangular box is 10 in. Its length increases at $2$ in./sec; its width decreases at $4$ in./sec. When the length is $8$ in. and the width is $6$ in., the rate (in $\text{in}^{3}/\text{sec}$) at which the volume of the box is changing is
\begin{enumerate}[label=(\Alph*)]
\item $200$ \item $80$ \item $-80$ \item $-200$
\end{enumerate}

% --- A43
\item The tangent to the curve $x^{3}+x^{2}y+4y=1$ at the point $(3,-2)$ has slope
\begin{enumerate}[label=(\Alph*)]
\item $-3$ \item $\dfrac{27}{13}$ \item $\dfrac{11}{9}$ \item $\dfrac{15}{13}$
\end{enumerate}

% --- A44
\item If $f(x)=ax^{4}+bx^{2}$ and $ab>0$, then
\begin{enumerate}[label=(\Alph*)]
\item the curve has no horizontal tangents
\item the curve is concave up for all $x$
\item the curve has no inflection point
\item none of the preceding is necessarily true
\end{enumerate}

% --- A45  (tables)
\item A function $f$ is continuous and differentiable on the interval $[0,4]$, where $f'>0$ but $f''$ is negative. Which table could represent points on $f$?
\medskip

\begin{tabular}{@{}l@{\quad}c@{\;}c@{\;}c@{\;}c@{\;}c@{}}
(A) &
$\begin{array}{c|ccccc}
x&0&1&2&3&4\\\hline
y&10&12&14&16&18
\end{array}$
&
\quad (B) &
$\begin{array}{c|ccccc}
x&0&1&2&3&4\\\hline
y&10&12&15&19&24
\end{array}$
&
\quad (C) &
$\begin{array}{c|ccccc}
x&0&1&2&3&4\\\hline
y&10&18&24&28&30
\end{array}$
\\[1.2em]
(D) &
$\begin{array}{c|ccccc}
x&0&1&2&3&4\\\hline
y&10&14&21&24&26
\end{array}$ &&&&
\end{tabular}

% --- A46
\item[*\textbf{A46.}] An equation of the tangent to the curve with parametric equations $x=2t+1$, $y=3-t^{3}$ at the point where $t=1$ is
\begin{enumerate}[label=(\Alph*)]
\item $2x+3y=12$ \item $3x+2y=13$ \item $6x+y=20$ \item $3x-2y=5$
\end{enumerate}

% --- A47
\item[*\textbf{A47.}] Given $f(x)=\sqrt[3]{x}$ and $f(64)=4$, approximately how much \emph{less} than $4$ is $\sqrt[3]{63}$ using the tangent line?
\begin{enumerate}[label=(\Alph*)]
\item $\dfrac{1}{48}$ \item $\dfrac{1}{16}$ \item $\dfrac{1}{3}$ \item $\dfrac{2}{3}$
\end{enumerate}

% --- A48
\item The best linear approximation for $f(x)=\tan x$ near $x=\tfrac{\pi}{4}$ is $y=$
\begin{enumerate}[label=(\Alph*)]
\item $1+\dfrac12\!\left(x-\tfrac{\pi}{4}\right)$
\item $1+\left(x-\tfrac{\pi}{4}\right)$
\item $1+\sqrt{2}\!\left(x-\tfrac{\pi}{4}\right)$
\item $1+2\!\left(x-\tfrac{\pi}{4}\right)$
\end{enumerate}

% --- A49
\item Given $f(x)=e^{kx}$, approximate $f(h)$ (for small $h$) using a tangent-line approximation. $f(h)\approx$
\begin{enumerate}[label=(\Alph*)]
\item $k$ \item $kh$ \item $1+k$ \item $1+kh$
\end{enumerate}

% --- A50 (parabola sketch)
\item If $f(x)=cx^{2}+dx+e$ for the function shown, then
\begin{center}
\begin{tikzpicture}[scale=0.95]
  \draw[->] (-0.5,0) -- (5.2,0) node[right] {$x$};
  \draw[->] (0,-1.2) -- (0,3.0) node[above] {$y$};
  \draw[domain=-0.1:4.6, smooth, very thick] plot (\x,{2.2 - 0.25*(\x+0.4)^2});
  \node at (3.8,2.3) {$f(x)$};
\end{tikzpicture}
\end{center}
\begin{enumerate}[label=(\Alph*)]
\item $c,d,e$ are all positive
\item $c>0,\ d<0,\ e>0$
\item $c<0,\ d>0,\ e>0$
\item $c<0,\ d<0,\ e>0$
\end{enumerate}

% --- A51
\item Given $f(x)=\log_{10}x$ and $\log_{10}(102)\approx 2.0086$, which is closest to $f'(100)$?
\begin{enumerate}[label=(\Alph*)]
\item $0.0043$ \item $0.0086$ \item $0.01$ \item $1.0043$
\end{enumerate}

\end{enumerate}

\end{document}

